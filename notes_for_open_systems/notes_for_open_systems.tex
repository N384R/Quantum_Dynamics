\documentclass{article}
\usepackage{lipsum}
\usepackage{qcircuit}
\usepackage{braket}

\title{\textbf{How to use \LaTeX ?}}
\author{\textbf{Jeyun Ju} \\ Department of Chemistry, KAIST}

\begin{document}
\maketitle

\begin{abstract}
This is a simple example to show how to use \LaTeX. We will learn how to write a simple document, how to use packages, and how to write mathematical equations. We will also learn how to use the \texttt{qcircuit} package to draw quantum circuits.
\end{abstract}

\section{Basic Quantum Circuits}
The quantum circuit for the Bell state $\ket{\Phi^+}$ is shown below:
\[
\Qcircuit @C=1em @R=1em {
& \gate{H} & \ctrl{1} & \qw & \meter \\
& \qw & \targ & \qw & \meter
}
\]


For an arbitrary gate, you can use the following:

\begin{center}
\begin{minipage}{0.3\textwidth}
\[
\Qcircuit @C=1em @R=1em {
& \gate{U} & \qw
}
\]
\end{minipage}
\begin{minipage}{0.3\textwidth}
\[
\Qcircuit @C=1em @R=1em {
& \multigate{1}{U} & \qw \\
& \ghost{U} & \qw
}
\]
\end{minipage}
\begin{minipage}{0.3\textwidth}
\[
\Qcircuit @C=1em @R=1em {
& \multigate{2}{U} & \qw \\
& \ghost{U} & \qw \\
& \ghost{U} & \qw
}
\]
\end{minipage}
\end{center}

\section{Mathematical Equations}

\section{Introduction}
\lipsum[2]

\end{document}